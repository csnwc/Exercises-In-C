\nsection{Sets}

Sets are an important concept in Computer Science. They enable the
storage of elements (members), guaranteeing that no element appears more than
once. Operations on sets include initializing them, copying them,
inserting an element, returning their size (cardinality), finding if
they contain a particular element, removing an element if it exists, and
removing one element from a random position (since sets have no particular
ordering, this could be the first element). Other set operations include
union (combining two sets to include all elements), and intersection
(the set containing elements common to both sets).

\wwwurl{https://www.mathsisfun.com/sets/sets-introduction.html}
\wwwurl{https://en.wikipedia.org/wiki/Set_(mathematics)}

The definition of a Set ADT is given in \verb^set.h^, and a file to test it is given
in \verb^testset.c^.

\begin{exercise}
Write an implementation of the Set ADT which builds on top of the Indexed Array ADT introduced in
Exercise~\ref{ex:indarray} and which compiles againset \verb^testset.c^.
\end{exercise}
