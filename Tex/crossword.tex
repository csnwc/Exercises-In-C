\nsection{Crosswords}
Many newspapers in the UK have a puzzles section. One of the most well
established such puzzle is the {\em crossword}.
\wwwurl{https://en.wikipedia.org/wiki/Crossword}
A crossword is a $2D$ grid of
full and empty cells. Clues are given so that the empty squares can be filled-in
with answers to {\em across} or {\em down} clues.

\begin{Puzzle}{5}{5}%
|[1]F|R |O   |[2]G|*   |.
|O   |* |*   |O   |*   |.
|N   |* |[3]A|D   |[4]D|.
|[5]T|U |G   |*   |A   |.
|*   |* |[6]E|L   |M   |.
\end{Puzzle}
\begin{PuzzleClues}{\textbf{Across}}
\Clue{1}{FROG}{Tailless amphibian}
\Clue{3}{ADD}{Join two items together}
\Clue{5}{TUG}{Pull something suddenly}
\Clue{6}{ELM}{Large deciduous tree}
\end{PuzzleClues}
\begin{PuzzleClues}{\textbf{Down}}
\Clue{1}{FONT}{Shape having one face and size}
\Clue{2}{GOD}{Higher being}
\Clue{3}{AGE}{Period of History}
\Clue{4}{DAM}{Water barrier}
\end{PuzzleClues}

\vspace*{1.5ex}
\noindent Here though, we aren't going to be dealing with solving crosswords, only in understanding
something about their structure.

\noindent Firstly, let us consider how crosswords are numbered.
In the above example the first square is shared by both $1$ across and $1$ down.
Working from the top-left rightwards and {\bf then downwards}, squares that are
at the start of a new word are numbered. The second square in the above example has no number
because it is already part of $1$ across, and there is no room for a down clue. 
On the fourth square, there is room for a down clue, but not for an across since it is already being used by $1$ down.

\noindent Actually, the algorithm for deciding on whether a square should be numbered or not is fairly simple:
\begin{enumerate}
\item All squares outside (off the grid) of the board are considered full.
\item The pattern:
\begin{Puzzle}{3}{1}%
|*|A |A |.
\end{Puzzle}
shows that we can start a new across clue from the middle square.
\item Similarly, the pattern:
\begin{Puzzle}{1}{3}%
|* |.
|A |.
|A |.
\end{Puzzle}
shows that we can start a new down clue from the middle square.
\item Sometimes a square can be the start of both an across and down clue.
In this case, they must share the same number.
\end{enumerate}

\noindent Secondly, another important concept in crossword clues is that of the number of {\em checked} squares.
This is the number of squares which are shared by both an across clue and a down clue. In the above crossword, nine squares
are shared of the $17$ empty ones which rounds to $53\%$. This is shown below:

\PuzzleDefineColorCell{c}{ocre}
\begin{Puzzle}{5}{5}%
|[1][c]F|R |O  |[2][c]G|* |.
|O |* |* |O |* |.
|N |* |[3][c]A|[][c]D |[4][c]D|.
|[5][c]T|U |[][c]G |* |A |.
|* |* |[6][c]E|L |[][c]M |.
\end{Puzzle}

\begin{exercise}
Complete the file {\bf crossword.c} which, along with my files {\em crossword.h} and {\em cwddriver.c},
allows a crossword to be created such that the clue numberings and percentage of checked squares
can be found. 

\noindent My file {\em crossword.h} contains the function definitions which you'll have to implement
in your {\bf crossword.c} file.
My file {\em cwddriver.c} contains the \verb^main()^ function to act as a driver to run the code.
Your file will contain many other functions as well as those specified, so you'll wish to test them
as normal using the \verb^test()^ function.

\noindent If all of these files are in the same directory, you can compile them using:
\begin{verbatim}
clang crossword.c cwddriver.c -o crossword -Wall -Wextra
-Wpedantic -std=c99 -Wvla -Wfloat-equal -O3 -Werror -lm
\end{verbatim}

\noindent Do not alter or resubmit {\em crossword.h} or {\em cwddriver.c} - my original versions will be used to compile the 
{\em crossword.c} file that you adapt.
\end{exercise}
