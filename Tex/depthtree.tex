\nsection{Depth}
\label{ex:randtree}

The following program builds a binary tree at random:
\begin{codesnippet}
#include <stdio.h>
#include <stdlib.h>
#include <assert.h>
#include <time.h>

#define STRSIZE 5000

struct node{
	char c;
	struct node *left;
	struct node *right;
};
typedef struct node Node;

Node *MakeNode(char c);
void InsertRandom(Node *t, Node *n);
char *PrintTree(Node *t);

int main(void)
{

   char c;
   Node *head = MakeNode('A');
   Node *n;

   srand(time(NULL));
   for(c = 'B'; c < 'G'; c++){
      n = MakeNode(c);
      InsertRandom(head, n);
   }
   printf("%s\n", PrintTree(head));
   return 0;
}

Node *MakeNode(char c)
{

   Node *n = (Node *)calloc(1, sizeof(Node));
   assert(n != NULL);
   n->c = c;
   return n;

}

void InsertRandom(Node *t, Node *n)
{

   if((rand()%2) == 0){ /* Left */
      if(t->left == NULL){
         t->left = n;
      }
      else{
         InsertRandom(t->left, n);
      }
   }
   else{ /* Right */
      if(t->right == NULL){
         t->right = n;
      }
      else{
         InsertRandom(t->right, n);
      }
   }

}

char *PrintTree(Node *t)
{

   char *str;

   assert((str = calloc(STRSIZE, sizeof(char))) != NULL);
   if(t == NULL){
      strcpy(str, "*");
      return str;
   }
   sprintf(str, "%c (%s) (%s)", t->c, PrintTree(t->left), PrintTree(t->right));
   return str;

}
\end{codesnippet}

Each node of the tree contains one of the characters 'A' $\ldots$ 'F'.
At the end, the tree is printed out in the manner described in the
course lectures.

\begin{exercise}
Adapt the code so that the maximum depth of the tree is computed using a recursive function.
The maximum depth of the tree is
the longest path from the root to a leaf. The depth of a tree
containing one node is $1$.

\end{exercise}
