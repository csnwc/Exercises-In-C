\nsection{Lisp and the CONS structure}

The name of the cons function is not unreasonable: it is an
abbreviation of the word “construct”. The origins of the names
for car and cdr, on the other hand, are esoteric: car is an acronym
from the phrase “Contents of the Address part of the Register”;
and cdr (pronounced “could-er”) is an acronym from the phrase
“Contents of the Decrement part of the Register”. These phrases
refer to specific pieces of hardware on the very early computer
on which the original Lisp was developed. Besides being obsolete,
the phrases have been completely irrelevant for more than 25 years to
anyone thinking about Lisp. Nonetheless, although a few brave scholars
have begun to use more reasonable names for these functions, the old
terms are still in use. In particular, since the terms are used in
the Emacs Lisp source code, we will use them in this introduction.
https://www.gnu.org/software/emacs/manual/html_node/eintr/Strange-Names.html



https://www.tutorialspoint.com/lisp/lisp_lists.htm

Atom: Number
Component : An atom or a list of atoms.
Lists : Sequence of components (i.e. each of which is an atom or another list)

car : It takes a list as argument, and returns the first component - this could be atomic, or a list.

cdr : It takes a list as argument, and returns a list without the first component 

cons : It takes two arguments, an element and a list and returns a list with the element inserted at the first place.
