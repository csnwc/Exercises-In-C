\nsection{NumMatch}


The game {\it Number Match} consists of a $2D$ rectangular grid of
numbers. The aim is to remove all of the numbers from the grid by finding
`matches' for pairs of numbers, which are either the same (e.g.~a $4$
with a $4$), or which total to ten (e.g.~a $3$ with a $7$).  Pairs also
need to be connected by being adjacent to each other ($8$-connected),
or else have no other numbers between them, so in the following grid:
\begin{center}
% 22812
% 68728
% 54759
% 94692
\begin{tikzpicture}
\matrix[matrix of nodes,nodes={draw=black, anchor=center, minimum size=.6cm,fill=ocre!30}, column sep=-\pgflinewidth, row sep=-\pgflinewidth, , execute at empty cell={\node[draw=black,text=black,fill=ocre!5]{.};} ] (A) {
2&2&8&1&2\\
6&8&7&2&8\\
5&4&7&5&9\\
9&4&6&9&2\\
};
\end{tikzpicture}
\end{center}
we could match a $7$ with a $7$:
\begin{center}
% 22812
% 68.28
% 54.59
% 94692
\begin{tikzpicture}
\matrix[matrix of nodes,nodes={draw=black, anchor=center, minimum size=.6cm,fill=ocre!30}, column sep=-\pgflinewidth, row sep=-\pgflinewidth, , execute at empty cell={\node[draw=black,text=black,fill=ocre!5]{.};} ] (A) {
2&2&8&1&2\\
6&8& &2&8\\
5&4& &5&9\\
9&4&6&9&2\\
};
\end{tikzpicture}
\end{center}
opening up the possibility of matching the $8$ with the $2$:
\begin{center}
% 22812
% 6...8
% 54.59
% 94692
\begin{tikzpicture}
\matrix[matrix of nodes,nodes={draw=black, anchor=center, minimum size=.6cm,fill=ocre!30}, column sep=-\pgflinewidth, row sep=-\pgflinewidth, , execute at empty cell={\node[draw=black,text=black,fill=ocre!5]{.};} ] (A) {
2&2&8&1&2\\
6& & & &8\\
5&4& &5&9\\
9&4&6&9&2\\
};
\end{tikzpicture}
\end{center}
If we now match the $4$ with the $4$:
\begin{center}
% 22812
% 6...8
% 5..59
% 9.692
\begin{tikzpicture}
\matrix[matrix of nodes,nodes={draw=black, anchor=center, minimum size=.6cm,fill=ocre!30}, column sep=-\pgflinewidth, row sep=-\pgflinewidth, , execute at empty cell={\node[draw=black,text=black,fill=ocre!5]{.};} ] (A) {
2&2&8&1&2\\
6& & & &8\\
5& & &5&9\\
9& &6&9&2\\
};
\end{tikzpicture}
\end{center}
then this frees up the `long' diagonal match of the $1$ with the $9$:
\begin{center}
% 228.2
% 6...8
% 5..59
% ..692
\begin{tikzpicture}
\matrix[matrix of nodes,nodes={draw=black, anchor=center, minimum size=.6cm,fill=ocre!30}, column sep=-\pgflinewidth, row sep=-\pgflinewidth, , execute at empty cell={\node[draw=black,text=black,fill=ocre!5]{.};} ] (A) {
2&2&8& &2\\
6& & & &8\\
5& & &5&9\\
 & &6&9&2\\
};
\end{tikzpicture}
\end{center}
If you are careful with the order in which you do the matching, it's possible to 
match all of the numbers.

\begin{exercise}
Write the functions specified in \verb^nm.h^ that allow a board to
be created by filling the board with random numbers
from $1 \ldots 9$ (\verb^randfill()^), \verb^take()^ to remove valid
pairs, and \verb^solve()^ which decides whether the board is solvable
or not.

\noindent You may assume that the height of the board is $4$ cells,
and the width is $5$.

\noindent
The brute-force algorithm for searching over all moves for a
solution goes like this~:
\begin{enumerate}
\item You will use a large array (list) of boards.
\item Put the initial board into the front of this list, \verb^f=0^.
\item Consider the board at the {\bf front} of the list (index \verb$f$).
\item For this (parent) board, find the resulting (child) boards 
which can be created from all the possible valid matches of number
pairs. For each of these child boards:
\begin{itemize}
\item If this board is unique (i.e.\ it has not been seen before in the list), add it to the end of the list.
\item If it has been seen before (it's a duplicate) ignore it.
\item If it is the `final' board, stop.
\end{itemize}
\item Add one to $f$. If there are more boards in the list, go to step $3$.
\end{enumerate}

\noindent
Your program~:
\begin{itemize}
\item {\bf Must} use the algorithm detailed above (which is similar to a queue and therefore a breadth-first search). Do not use the other algorithms possible (e.g. best-first, guided, recursive etc.); the quality of your coding is being assessed against others taking the same approach, and if you do something different it won't get any marks.
\item {\bf Must not} use dynamic arrays or linked lists.
The list of boards can be a fixed (large) size ($100,000$?)
\item {\bf Should not} print anything out to screen after successfully
completing the search.
\item {\bf Should} call implement the function \verb^test()^ to perform any assertion testing, (including tests of any of your helper functions).
\item Should run silently (except in the case of an \verb^EXIT_FAILURE^).
\end{itemize}


\subsection*{Extension}

Basic assignment = {\Large $90\%$}.
Extension = {\Large $10\%$}.

\noindent
{\bf If} you get your program working, there's a small extension available.
Recode your program so that it executes as quickly as possible in files
called \verb^ext.c^ and \verb^ext_mydefs.h^.
For this I will compute the time taken for your code to solve
a series of complex boards using \verb^/usr/bin/time^.
\end{exercise}

