\nsection{Rule 110}

\tikzset{word ladder/.style={
  matrix of nodes
  , execute at empty cell={\node[draw=gray,text=gray,fill=white]{0};}
  , nodes in empty cells=false
  , nodes={shape=rectangle, draw=none,fill=none,text=black,minimum width=0.6cm,minimum height=0.35cm,outer sep=0pt,align=center,inner sep=0pt,font=\small}
  , row sep={0.35cm,between origins}
  , column sep={0.6cm,between origins}
},
}

\tikzset{spacehash board/.style={
  matrix of nodes
  , execute at empty cell={\node[draw=gray,text=gray,fill=white]{ };}
  , nodes in empty cells=false
  , nodes={draw=gray,fill=ocre,text=gray,text depth=0.5ex,text height=2ex,text width=1em,outer sep=0pt,align=center,inner sep=0pt}
  , row sep={#1,between origins}
  , column sep={#1,between origins}
},
}

\tikzset{noughtsone board/.style={
  matrix of nodes
  , execute at empty cell={\node[draw=gray,text=gray,fill=white]{0};}
  , nodes in empty cells=false
  , nodes={draw=gray,fill=ocre,text=gray,minimum width=#1,minimum height=#1,outer sep=0pt,align=center,inner sep=0pt}
  , row sep={#1,between origins}
  , column sep={#1,between origins}
},
  noughtsone board/.default=0.5cm
}

\tikzset{bingrid board/.style={
  matrix of nodes
  , execute at empty cell={\node[draw=black,text=black,fill=white]{.};}
  , nodes in empty cells=false
  , nodes={draw=black,fill=white,text=black,minimum width=#1,minimum height=#1,outer sep=0pt,align=center,inner sep=0pt}
  , row sep={#1,between origins}
  , column sep={#1,between origins}
},
  bingrid board/.default=0.75cm
}

\tikzset{twocolour board/.style={
  matrix of nodes
  , execute at empty cell={\node[text=white,fill=white]{+};}
  , nodes in empty cells=false
  , nodes={draw=gray,fill=ocre,minimum width=#1,minimum height=#1,outer sep=0pt,align=center,inner sep=0pt,font=\tiny}
  , text=ocre
  , row sep={#1,between origins}
  , column sep={#1,between origins}
},
  twocolour board/.default=0.5cm
}
\tikzset{twocolour board/.style={
  matrix of nodes
  , execute at empty cell={\node[text=white,fill=white]{+};}
  , nodes in empty cells=false
  , nodes={draw=gray,fill=ocre,minimum width=#1,minimum height=#1,outer sep=0pt,align=center,inner sep=0pt,font=\tiny}
  , text=ocre
  , row sep={#1,between origins}
  , column sep={#1,between origins}
},
  twocolour board/.default=0.5cm
}

%% Candy Crush-style games, colour & label
\tikzset{crushstyle board/.style={
  matrix of nodes
  , nodes={draw=gray,fill=ocre,minimum width=#1,minimum height=#1,outer sep=0pt,align=center,inner sep=0pt,font=\tiny}
  , text=ocre
  , row sep={#1,between origins}
  , column sep={#1,between origins}
},
  crushstyle board/.default=0.5cm
}

%% WaTor-style games, colour & label
\tikzset{watorstyle board/.style={
  matrix of nodes
  , nodes={fill=cyan,minimum width=#1,minimum height=#1,outer sep=0pt,align=center,inner sep=0pt,font=\tiny}
  , text=black
  , row sep={#1,between origins}
  , column sep={#1,between origins}
},
  watorstyle board/.default=0.25cm
}

%% 8-tile style
\tikzset{eighttilestyle board/.style={
  matrix of nodes, ampersand replacement={\&}
  , execute at empty cell={\node[draw=gray,text=gray,fill=gray]{0};}
  , nodes={fill=gray,minimum width=#1,minimum height=#1,outer sep=0pt,align=center,inner sep=0pt,font=\tiny}
  , text=ocre
  , row sep={#1,between origins}
  , column sep={#1,between origins}
},
  eighttilestyle board/.default=0.25cm
}

\tikzset{sixelstyle/.style={
  matrix of nodes, ampersand replacement={\&}
  , nodes={draw=black,fill=gray,text=gray
     %,minimum height=1pt, minimum width=1pt
     %,row sep=1pt, column sep=1pt
}
},
   sixelstyle/.default=0.3em
}

\tikzset{sepsixstyle/.style={
  matrix of nodes, ampersand replacement={\&}
  , nodes={draw=white,fill=gray,text=gray,
     minimum height=1pt, minimum width=1pt,
     row sep=1pt, column sep=1pt}
},
   sixelstyle/.default=0.3em
}


Rather interesting patterns can be created using
{\it Cellular Automata}. Here we will use a simple
example, one known as {\it Rule 110}~:
The idea is that in a 1D array, cells can be either
on
\begin{tikzpicture}
\draw[fill=ocre,draw=gray] (0,0) rectangle (2ex,2ex);
\end{tikzpicture}
or off
\begin{tikzpicture}
\draw[fill=none,draw=gray] (0,0) rectangle (2ex,2ex);
\end{tikzpicture}
(perhaps represented by the integer values
$1$ and $0$).
A new 1D array is created in which we decide upon the
state of each cell in the array based on the cell above
and its two immediate neighbours.

If the three cells above are all `on', then the cell is set
to `off' ($111 \rightarrow 0$). If the three cells above
are `on', `on', `off' then the new cell is set to `on' ($110 \rightarrow 1$).
The rules, in full, are:\\
$111 \rightarrow 0$\\
$110 \rightarrow 1$\\
$101 \rightarrow 1$\\
$100 \rightarrow 0$\\
$011 \rightarrow 1$\\
$010 \rightarrow 1$\\
$001 \rightarrow 1$\\
$000 \rightarrow 0$\\

You take a 1D array, filled with zeroes or ones,
and based on these, you create a new 1D array of
zeroes and ones. Any particular cell uses the three
cells `above' it to make the decision about its
value. If the first line has all zeroes
and a single one in the middle, then the automata evolves as:
\begin{figure}[ht]
\centering
\begin{tikzpicture}
{\fontsize{5}{5}\selectfont
\matrix [noughtsone board=0.375cm]
{
& & & & & & & & & & & & & & & & & & & & & & & & & & & & & &1& & &\\
& & & & & & & & & & & & & & & & & & & & & & & & & & & & &1&1& & &\\
& & & & & & & & & & & & & & & & & & & & & & & & & & & &1&1&1& & &\\
& & & & & & & & & & & & & & & & & & & & & & & & & & &1&1& &1& & &\\
& & & & & & & & & & & & & & & & & & & & & & & & & &1&1&1&1&1& & &\\
& & & & & & & & & & & & & & & & & & & & & & & & &1&1& & & &1& & &\\
& & & & & & & & & & & & & & & & & & & & & & & &1&1&1& & &1&1& & &\\
& & & & & & & & & & & & & & & & & & & & & & &1&1& &1& &1&1&1& & &\\
& & & & & & & & & & & & & & & & & & & & & &1&1&1&1&1&1&1& &1& & &\\
& & & & & & & & & & & & & & & & & & & & &1&1& & & & & &1&1&1& & &\\
& & & & & & & & & & & & & & & & & & & &1&1&1& & & & &1&1& &1& & &\\
& & & & & & & & & & & & & & & & & & &1&1& &1& & & &1&1&1&1&1& & &\\
& & & & & & & & & & & & & & & & & &1&1&1&1&1& & &1&1& & & &1& & &\\
& & & & & & & & & & & & & & & & &1&1& & & &1& &1&1&1& & &1&1& & &\\
& & & & & & & & & & & & & & & &1&1&1& & &1&1&1&1& &1& &1&1&1& & &\\
& & & & & & & & & & & & & & &1&1& &1& &1&1& & &1&1&1&1&1& &1& & &\\
& & & & & & & & & & & & & &1&1&1&1&1&1&1&1& &1&1& & & &1&1&1& & &\\
& & & & & & & & & & & & &1&1& & & & & & &1&1&1&1& & &1&1& &1& & &\\
& & & & & & & & & & & &1&1&1& & & & & &1&1& & &1& &1&1&1&1&1& & &\\
& & & & & & & & & & &1&1& &1& & & & &1&1&1& &1&1&1&1& & & &1& & &\\
& & & & & & & & & &1&1&1&1&1& & & &1&1& &1&1&1& & &1& & &1&1& & &\\
& & & & & & & & &1&1& & & &1& & &1&1&1&1&1& &1& &1&1& &1&1&1& & &\\
& & & & & & & &1&1&1& & &1&1& &1&1& & & &1&1&1&1&1&1&1&1& &1& & &\\
& & & & & & &1&1& &1& &1&1&1&1&1&1& & &1&1& & & & & & &1&1&1& & &\\
& & & & & &1&1&1&1&1&1&1& & & & &1& &1&1&1& & & & & &1&1& &1& & &\\
& & & & &1&1& & & & & &1& & & &1&1&1&1& &1& & & & &1&1&1&1&1& & &\\
& & & &1&1&1& & & & &1&1& & &1&1& & &1&1&1& & & &1&1& & & &1& & &\\
& & &1&1& &1& & & &1&1&1& &1&1&1& &1&1& &1& & &1&1&1& & &1&1& & &\\
& &1&1&1&1&1& & &1&1& &1&1&1& &1&1&1&1&1&1& &1&1& &1& &1&1&1& & &\\
&1&1& & & &1& &1&1&1&1&1& &1&1&1& & & & &1&1&1&1&1&1&1&1& &1& & &\\
};
}
\end{tikzpicture}
\caption{1D cellular automaton using Rule 110. Top line shows initial state,
each subsequent line is produced from the line above it. Each cell has
a rule to switch it `on' or 'off' based on the state of the three cells
above it in the diagram.}
\label{rulle110_fig}
\end{figure}

\begin{exercise} Write a program that outputs something similar to
Figure~\ref{rulle110_fig}, but using plain text, giving the user
the option to start with a randomised first line, or a line with a
single `on' in the central location. Do not use $2D$ arrays for this -
a couple of $1D$ arrays is sufficient.  \end{exercise}

\begin{exercise}
Rewrite the program above to allow other rules to be displayed - for instance $124, 30$ and $90$.
\wwwurl{http://en.wikipedia.org/wiki/Rule_110}
\end{exercise}
