\nsection{Randomness}
The function \verb^rand()^ returns values in the interval [0, RAND\_MAX].
If we declare the variable \verb^median^ and initialise it to have the
value \verb^RAND_MAX/2^, then \verb^rand()^ will return a value
that is sometimes larger than \verb^median^ and sometimes smaller.

\begin{exercise}
Write a program that calls \verb^rand()^, say $500$ times,
inside a \verb^for^ loop, increments the variable \verb^minus_cnt^
every time \verb^rand()^ returns a value less than \verb^median^.
Each time through the \verb^for^ loop, print out the value of the difference
of \verb^plus_cnt^ and \verb^minus_cnt^.
You might think that this difference should oscillate near zero. Does it~?
\end{exercise}`
