\nsection{Hailstone Sequence}

Hailstones sequences are ones that seem to
always return to $1$. The number is halved if even, and if
odd then the next becomes \verb^3*n+1^. For instance, when we start
with the number $6$, we get the sequence~:
\verb^6, 3, 10, 5, 16, 8, 4, 2, 1^
that has nine numbers in it. When we start with
the number $11$, the sequence is longer, containing $15$
numbers~:
\verb^11, 34, 17, 52, 26, 13, 40, 20, 10, 5, 16, 8, 4, 2, 1^.

\noindent {\bf Note} that whilst the sequences tend to be relatively short, hailstone numbers can be extremely large.
You may need a storage type larger than an {\em int} for storing them.

\begin{exercise}
Write a program that~:
\begin{itemize}
\item displays which initial number (less than
$10,000,000$) creates the {\bf longest} hailstone sequence.
\end{itemize}
\end{exercise}

\begin{exercise}
Write a program that~:
\begin{itemize}
\item displays which initial number (less than
$10,000,000$) leads to the {\bf largest} number appearing in the sequence.
\end{itemize}
\end{exercise}
