\renewcommand{\A}{|[fill=black,text=white]|A}
\renewcommand{\B}{|[fill=black,text=ocre]|B}
\renewcommand{\C}{|[fill=black,text=green]|C}

\nsection{Match Drop}

The puzzle {\it Match Drop} consists of a $2D$ rectangular grid of
tiles, all of which are labelled with an upper-case letter e.g. `$A$', `$B \ldots$ `$Z$'. Outside of this grid is another tile, known as the
`hawk' tile:

\begin{tikzpicture}[every node/.style={anchor=base,text depth=.5ex,text height=2ex,text width=1em,outer sep=0pt,align=center,inner sep=0pt}] \matrix [matrix of nodes,draw=white,nodes in empty cells] {
\A&  &  \\
\A&\B&\C\\
\A&\B&\C\\
\C&\B&\A\\
};
\end{tikzpicture}

\noindent The 'hawk' tile can be used to push down one column.
The hawk becomes the top tile in this column, and the bottom tile of
this column becomes the new hawk tile. The task is to roll one column
at a time, so that every column contains the same letters.

\noindent
In the above example, if the hawk tile is played to push down the first
column, then the new board now looks like:

\begin{tikzpicture}[every node/.style={anchor=base,text depth=.5ex,text height=2ex,text width=1em,outer sep=0pt,align=center,inner sep=0pt}] \matrix [matrix of nodes,draw=white,nodes in empty cells] {
\C&  &  \\
\A&\B&\C\\
\A&\B&\C\\
\A&\B&\A\\
};
\end{tikzpicture}

\noindent
Both the first and seond columns are now completed and are never
altered again. Using the hawk on column three produces:

\begin{tikzpicture}[every node/.style={anchor=base,text depth=.5ex,text height=2ex,text width=1em,outer sep=0pt,align=center,inner sep=0pt}] \matrix [matrix of nodes,draw=white,nodes in empty cells] {
\A&  &  \\
\A&\B&\C\\
\A&\B&\C\\
\A&\B&\C\\
};
\end{tikzpicture}

\noindent
and our search for a finshed (completed) board is over.


\begin{exercise}
Write the functions specified in \verb^md.h^ that allows a board file
to be read in, and computes the number of moves required to solve it.

\noindent
You may assume that the maximum height and width of a board is $6$.

\noindent
The brute-force algorithm for searching over all moves for a
solution goes like this~:
\begin{enumerate}
\item You will use an \verb^alloc()^'d array (list) of boards.
\item Put the initial board into the front of this list, \verb^f=0^.
\item Consider the board at the {\bf front} of the list (index \verb$f$).
\item For this (parent) board, find the resulting (child) boards 
which can be created from all the possible column pushes (already
completed columns are not altered). For each of these child boards:
\begin{itemize}
\item If this board is unique (i.e.\ it has not been seen before in the list), add it to the end of the list.
\item If it has been seen before (it's a duplicate) ignore it.
\item If it is the `final' board, stop and (possibly, print the solution).
\end{itemize}
\item Add one to $f$. If there are more boards in the list, go to step $3$.
\end{enumerate}








\noindent To help with printing out the correct moves, when a solution
has been found, each board in the list will need to contain (amongst
other things) a 2D grid of tiles, the hawk, and a record of its parent board, i.e. the board
that it was created from. Since you're using an array, this could simply
be the index of the array that was the parent.

\noindent
Your program~:
\begin{itemize}
\item {\bf Must} use the algorithm detailed above (which is similar to a queue and therefore a breadth-first search). Do not use the other algorithms possible (e.g. best-first, guided, recursive etc.); the quality of your coding is being assessed against others taking the same approach, and if you do something different it won't get any marks.
\item {\bf Must not} use dynamic arrays or linked lists. Since boards cannot be any larger than $6 \times 6$, you can create boards of this size, and only use a sub-part of them if the board required is smaller. The list of boards can be a fixed (large) size (maybe $200,000$?)
\item {\bf Should} be able to cope with invalid board definition files with a graceful exit.
\item {\bf Should not} print anything out to screen after successfully
completing the search, except when in verbose mode. Automated checking
will be used during marking, and therefore the output must be very precise.
For the \verb^driver.c^ file given, the verbose output is required for
\verb^2moves.brd^, for which the verbose flag has been set in the 
solve function. In this case, the output will look like:
\begin{verbatim}
% ./md
ABC
ABC
ABC
CBA

ABC
ABC
ABC
ABA

ABC
ABC
ABC
ABC

\end{verbatim}

\item {\bf Should} call the function \verb^test()^ to perform any assertion testing etc.
\end{itemize}

\end{exercise}
